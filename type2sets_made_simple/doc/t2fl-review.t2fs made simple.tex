\documentclass[]{article}

\usepackage{graphicx}
\usepackage{listings}
\usepackage[table]{xcolor}% http://ctan.org/pkg/xcolor

%opening
\title{Type2Fuzzy Library Implementation: Mendel, Jerry M., and RI Bob John. "Type-2 fuzzy sets made simple." IEEE Transactions on fuzzy systems 10.2 (2002): 117-127.}
\author{Carmel Gafa}

\begin{document}

\maketitle

\begin{abstract}
"Type-2 Fuzzy Sets made simple" is possibly the best paper to learn about Type-2 fuzzy sets and logic. It outlines all the definitions and concepts that are necessary to work with type-2 fuzzy sets in a clear and concise manner. This paper illustrates the implementation of all the examples prepared by Mendel and John using the type2fuzzy library.
\end{abstract}

\section{Introduction}
This paper is the first in a series aimed to illustrate the capabilities of the Type2FuzzyLibrary (https://pypi.org/project/type2fuzzy/) This is achieved by working up the numerical examples in selected papers using the library and comparing the results with those obtained by the original authors. The papers will list the code used to carry out the examples, and the results obtained. All code is written using the Python language.


\section{Type-2 fuzzy set definition}

The paper illustrates several type-2 fuzzy sets concepts with a simple general type-2 fuzzy set,

\bigskip

{\small (0.9/0 + 0.8/0.2+ 0.7/0.4 + 0.6/0.6 + 0.5/0.8)/1}\\
{\small +(0.5/0 + 0.35/0.2 + 0.35/0.4 + 0.2/0.6 + 0.5/0.8)/2}\\
{\small +(0.35/0.6 + 0.35/0.8)/3}\\
{\small +(0.1/0 + 0.35/0.2 + 0.5/0.4 + 0.1/0.6 + 0.35/0.8)/4}\\
{\small +(0.35/0 + 0.5/0.2 + 0.1/0.4 + 0.2/0.6 + 0.2/0.8)/5}\\

This set will be used in this exercise as in the paper.

\bigskip

A type-2 fuzzy set $\tilde{A}$ can be expressed as 

\begin{equation}
\tilde{A}=\int\displaylimits_{x\in X}\int\displaylimits_{u\in J_{x}} \mu_{\tilde{A}}(x,u) \slash (x,u)
\end{equation} 

where $J_{x}\subseteq[0,1]$

\bigskip

The following code snippet illustrates how a general type-2 fuzzy set is defined and used, as explained in Example 1 of the original paper. 

\bigskip

\begin{tabular}{|l|}
\hline 
\lstset{language=Python}
\lstset{basicstyle=\scriptsize}
\begin{lstlisting}
from type2fuzzy import GeneralType2FuzzySet

'''
Example 1 : definition of the general type-2 fuzzy set
'''

gt2fs_rep =   ''' (0.9/0 + 0.8/0.2+ 0.7/0.4 + 0.6/0.6 + 0.5/0.8)/1
+(0.5/0 + 0.35/0.2 + 0.35/0.4 + 0.2/0.6 + 0.5/0.8)/2
+(0.35/0.6 + 0.35/0.8)/3
+(0.1/0 + 0.35/0.2 + 0.5/0.4 + 0.1/0.6 + 0.35/0.8)/4
+(0.35/0 + 0.5/0.2 + 0.1/0.4 + 0.2/0.6 + 0.2/0.8)/5'''

# create set
gt2fs = GeneralType2FuzzySet.from_representation(gt2fs_rep)
print(f'\nSet representation: {gt2fs}')
\end{lstlisting}
\\
\hline
\\
Set representation:
{\small (0.9000 / 0.0000 + 0.8000 / 0.2000 + 0.7000 / 0.4000 + }\\
{\small 0.6000 / 0.6000 + 0.5000 / 0.8000) / 1.0000}\\
{\small + (0.5000 / 0.0000 + 0.3500 / 0.2000 + 0.3500 / 0.4000 + }\\
{\small 0.2000 / 0.6000 + 0.5000 / 0.8000) / 2.0000 }\\
{\small + (0.3500 / 0.6000 + 0.3500 / 0.8000) / 3.0000 + (0.1000 / 0.0000}\\
{\small + 0.3500 / 0.2000 + 0.5000 / 0.4000 +}\\
{\small 0.1000 / 0.6000 + 0.3500 / 0.8000) / 4.0000 +} \\
{\small (0.3500 / 0.0000 + 0.5000 / 0.2000 + 0.1000 / 0.4000 +} \\
{\small 0.2000 / 0.6000 + 0.2000 / 0.8000) / 5.0000}\\
\\ 
\hline 
\end{tabular} 


\section{Vertical Slice}

 A vertical slice is Type-1 fuzzy set $\mu_{\tilde{A}}(x=x',u)$ for $x\in X$ and $\forall u \in J_{x'}\subseteq[0,1]$, that is:
\begin{equation}
\mu_{\tilde{A}}(x=x',u)=\int\displaylimits_{u\in J_{x'}}f_{x'}(u)\slash u
\end{equation} 
where $0\leq f_{x'}(u)\leq 1$

\bigskip

The following code snippet illustrates two methods by which a vertical slice can be obtained to replicate the second part of Example 1.

\bigskip

\begin{tabular}{|l|}
\hline 
\lstset{language=Python}
\lstset{basicstyle=\scriptsize}
\\
\begin{lstlisting}
# different ways to get vertical slice
print('mu_a_tilde(',1,')= ', gt2fs.vertical_slice(1))
print('mu_a_tilde(',2,')= ', gt2fs[2])
print('mu_a_tilde(',3,')= ', gt2fs.vertical_slice(3))
print('mu_a_tilde(',4,')= ', gt2fs[4])
\end{lstlisting}
\\
\hline
\\
{\small mu( 1 )= 0.900/0.000 + 0.800/0.200 + 0.700/0.400 + 0.600/0.600 + 0.500/0.800}\\
{\small mu( 2 )=0.500/0.000 + 0.350/0.200 + 0.350/0.400 + 0.200/0.600 + 0.500/0.800}\\
{\small mu( 3 )=0.350/0.600 + 0.350/0.800}\\
{\small mu( 4 )=0.100/0.000 + 0.350/0.200 + 0.500/0.400 + 0.100/0.600 + 0.350/0.800}\\
\\ 
\hline 
\end{tabular} 

\section{Primary Membership}
The \textbf{domain} of a secondary membership function is called the \textbf{primary membership} of $x$. Hence in
\[
\tilde{A}=\int\displaylimits_{x \in X} \mu_{ \tilde{A} }(x) /x = \int\displaylimits_{x \in X} \left[  \int\displaylimits_{u\in J_{x}} f(u) / u \right]  /x
\]
$J_{x}$ is the primary membership function, where $J_{x} \subseteq [0,1]$ for $\forall x \in X$

\bigskip

The code below illustrates the final part of Example 1 where the primary memberships of the general type-2 fuzzy set are listed:

\bigskip


\begin{tabular}{|l|}
\hline 
\lstset{language=Python}
\lstset{basicstyle=\scriptsize}
\\
\begin{lstlisting}
# get the primary memberships of the set
# example 1 (continued)
print('\nPrimary Membership:')
for x_k in gt2fs.primary_domain():
print('J_',x_k, ' : ',  gt2fs.primary_membership(x_k)) 
\end{lstlisting}
\\
\hline
\\
{\small Primary Membership:}\\
{\small J1.0  :  [0.0, 0.2, 0.4, 0.6, 0.8]}\\
{\small J2.0  :  [0.0, 0.2, 0.4, 0.6, 0.8]}\\
{\small J3.0  :  [0.6, 0.8]}\\
{\small J4.0  :  [0.0, 0.2, 0.4, 0.6, 0.8]}\\
{\small J5.0  :  [0.0, 0.2, 0.4, 0.6, 0.8]}\\	
\\
\hline 
\end{tabular} 


\section{Secondary Grade}
The \textbf{amplitude} of a secondary membership function is the \textbf{secondary grade}. Hence in
\[
\tilde{A}=\int_{x \in X} \mu_{ \tilde{A} }(x) /x = \int_{x \in X} \left[  \int_{u\in J_{x}} f(u) / u \right]  /x
\]
where $J_{x} \subseteq [0,1]$, $f(u)$ is the secondary grade.

\bigskip

The following code illustrates the retrieval of selected secondary grade values from the general type-2 fuzzy set

\bigskip

\begin{tabular}{|l|}
\hline 
\lstset{language=Python}
\lstset{basicstyle=\scriptsize}
\begin{lstlisting}
# get the secondary grade of some values
# example 1 (continued)
print('\nSecondary grade of some points:')
print('mu(1,0.2)=',  gt2fs.secondary_grade(1, 0.2), '-- should be 0.8') 
print('mu(2,0)=',  gt2fs.secondary_grade(2, 0), '-- should be 0.5') 
print('mu(3,0.8)=',  gt2fs.secondary_grade(3, 0.8), '-- should be 0.35') 
print('mu(4,0.4)=',  gt2fs.secondary_grade(4, 0.4), '-- should be 0.5') 
\end{lstlisting}
\\
\hline
\\
{\small Secondary grade of some points:}\\
{\small mu(1,0.2)= 0.8 -- should be 0.8}\\
{\small mu(2,0)= 0.5 -- should be 0.5}\\
{\small mu(3,0.8)= 0.35 -- should be 0.35}\\
{\small mu(4,0.4)= 0.5 -- should be 0.5}\\
\\
\hline 
\end{tabular} 


\section{Footprint of Uncertainty}

The 2D support of $\mu$ is called \textbf{the footprint of uncertainty (FOU)}

\begin{equation}
FOU(\tilde{A})= \left\lbrace  (x,u) \in X \times [0,1]  | \mu_{\tilde{A}}(x,u) > 0 \right\rbrace 
\end{equation}

FOU represents the uncertainty in the primary memberships of $\tilde{A}$. It is the union of all primary memberships 
\begin{equation}
FOU(\tilde{A}) = \bigcup\limits_{x\in X} J_{x}
\end{equation}

\bigskip

The FOU can be retrieved using a single line of type2fuzzy library code;

\bigskip

\begin{tabular}{|l|}
\hline 
\lstset{language=Python}
\lstset{basicstyle=\scriptsize}
\begin{lstlisting}
# get the footprint of uncertainty for the set
footprint = gt2fs.footprint_of_uncertainty()
print('\nFootprint of uncertainty: ', footprint)
\end{lstlisting}
\\
\hline
\\
{\small Footprint of uncertainty:  \{}\\
{\small 1.0: CrispSet([0.00000, 0.80000]), }\\
{\small 2.0: CrispSet([0.00000, 0.80000]), }\\
{\small 3.0: CrispSet([0.60000, 0.80000]), }\\
{\small 4.0: CrispSet([0.00000, 0.80000]), }\\
{\small 5.0: CrispSet([0.00000, 0.80000])\}}\\
\\
\hline 
\end{tabular} 


\section{Embedded Type-2 Fuzzy Sets}

For discrete universes of discourse $X$ and $U$, an \textbf{embedded type-2 set} $\tilde{A_e}$ has $N$ elements, where $\tilde{A_e}$ has exactly one element from $J_{x_{1}}, J_{x_{2}}, \dots , J_{x_{N}}$; namely $u_{1}, u_{2}, \dots , u_{N}$ each with associated grade namely $f_{x_{1}}(u_1), f_{x_{2}}(u_2), \dots , f_{x_{N}}(u_N)$, such that:

\begin{equation}
\tilde{A_e} = \displaystyle \sum_{i=1}^{N} \left[ f_{x_{i}} (u_{i}) \right] / x_{i}
\end{equation}

where $u_{i} \in J_{x_{i}} \subseteq [0,1]$

Set $\tilde{A_e}$ is embedded in $\tilde{A}$ and there are a total of:

\begin{equation}
Num(\tilde{A_e}) = \displaystyle \prod_{i=1}^{N} M_i
\end{equation}

\bigskip

In Example 2, the authors depict one of the possible 1250 embedded type-2 fuzzy sets that are possible from the general type-2 fuzzy set. Two examples are presented here:
\begin{itemize}
	\item The first illustrates the method to obtain the number of embedded type-2 fuzzy sets.
	\item The second shows how the embedded type-2 fuzzy sets can be listed.
\end{itemize}

\bigskip

\begin{tabular}{|l|}
\hline 
\lstset{language=Python}
\lstset{basicstyle=\scriptsize}
\begin{lstlisting}
# number of embedded sets
print('\nNumber of embedded type-2 sets: ', 
gt2fs.embedded_type2_sets_count())
\end{lstlisting}
\\
\hline
{\small Number of embedded type-2 sets:  1250}\\
\hline 
\end{tabular} 

\bigskip

\begin{tabular}{|l|}
\hline 
\lstset{language=Python}
\lstset{basicstyle=\scriptsize}
\begin{lstlisting}
# Example 2
# list all embedded sets
count = 0
print('\nShowing first 10 embedded sets:')
for embedded_set in gt2fs.embedded_type2_sets():
print(embedded_set)
count = count+1
if count > 10:
break
\end{lstlisting}
\\
\hline
\\
{\small Showing first 10 embedded sets:}\\
{\small [(0.9, 0.0, 1.0), (0.5, 0.0, 2.0), (0.35, 0.6, 3.0), (0.1, 0.0, 4.0), (0.35, 0.0, 5.0)]}\\
{\small [(0.9, 0.0, 1.0), (0.5, 0.0, 2.0), (0.35, 0.6, 3.0), (0.1, 0.0, 4.0), (0.5, 0.2, 5.0)]}\\
{\small [(0.9, 0.0, 1.0), (0.5, 0.0, 2.0), (0.35, 0.6, 3.0), (0.1, 0.0, 4.0), (0.1, 0.4, 5.0)]}\\
{\small [(0.9, 0.0, 1.0), (0.5, 0.0, 2.0), (0.35, 0.6, 3.0), (0.1, 0.0, 4.0), (0.2, 0.6, 5.0)]}\\
{\small [(0.9, 0.0, 1.0), (0.5, 0.0, 2.0), (0.35, 0.6, 3.0), (0.1, 0.0, 4.0), (0.2, 0.8, 5.0)]}\\
{\small [(0.9, 0.0, 1.0), (0.5, 0.0, 2.0), (0.35, 0.6, 3.0), (0.35, 0.2, 4.0), (0.35, 0.0, 5.0)]}\\
{\small [(0.9, 0.0, 1.0), (0.5, 0.0, 2.0), (0.35, 0.6, 3.0), (0.35, 0.2, 4.0), (0.5, 0.2, 5.0)]}\\
{\small [(0.9, 0.0, 1.0), (0.5, 0.0, 2.0), (0.35, 0.6, 3.0), (0.35, 0.2, 4.0), (0.1, 0.4, 5.0)]}\\
{\small [(0.9, 0.0, 1.0), (0.5, 0.0, 2.0), (0.35, 0.6, 3.0), (0.35, 0.2, 4.0), (0.2, 0.6, 5.0)]}\\
{\small [(0.9, 0.0, 1.0), (0.5, 0.0, 2.0), (0.35, 0.6, 3.0), (0.35, 0.2, 4.0), (0.2, 0.8, 5.0)]}\\
{\small [(0.9, 0.0, 1.0), (0.5, 0.0, 2.0), (0.35, 0.6, 3.0), (0.5, 0.4, 4.0), (0.35, 0.0, 5.0)]}\\
\\
\hline 
\end{tabular} 

\bigskip

In the original paper, Example 3 considers the following general type-2 fuzzy set:

\[
(0.5/0.9)/x_1 + (0.2/0.7)/x_1 + (0.9/0.2)/x_1 + (0.6/0.6)/x_2 + (0.1/0.4)/x_2
\]

For the sake of this exercise, we assign the values of $x_1 = 1$ and $x_2=2$

thus obtaining the following set;

\[
(0.5/0.9)/1 + (0.2/0.7)/1 + (0.9/0.2)/1 + (0.6/0.6)/2 + (0.1/0.4)/2
\]

The embedded type-2 fuzzy sets are listed using the code below;

\bigskip

\begin{tabular}{|l|}
	\hline 
	\lstset{language=Python}
	\lstset{basicstyle=\scriptsize}
	\begin{lstlisting}
# Example 3
print('\nEmbedded set listing for general type-2 fuzzy set')
print(str(gt2fs_2))
for embedded_set in gt2fs_2.embedded_type2_sets():
print(embedded_set)
	\end{lstlisting}
	\\
	\hline
{\small Embedded set listing for general type-2 fuzzy set}\\
{\small (0.5000 / 0.9000 + 0.2000 / 0.7000 + 0.9000 / 0.2000) / 1.0000 }\\
{\small + (0.6000 / 0.6000 + 0.1000 / 0.4000) / 2.0000}\\
{\small [(0.9, 0.2, 1.0), (0.1, 0.4, 2.0)]}\\
{\small [(0.9, 0.2, 1.0), (0.6, 0.6, 2.0)]}\\
{\small [(0.2, 0.7, 1.0), (0.1, 0.4, 2.0)]}\\
{\small [(0.2, 0.7, 1.0), (0.6, 0.6, 2.0)]}\\
{\small [(0.5, 0.9, 1.0), (0.1, 0.4, 2.0)]}\\
{\small [(0.5, 0.9, 1.0), (0.6, 0.6, 2.0)]}\\
	\hline 
\end{tabular} 

\end{document}








