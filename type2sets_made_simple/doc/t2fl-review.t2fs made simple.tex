\documentclass[]{article}

\usepackage{graphicx}
\usepackage{listings}
\usepackage[table]{xcolor}% http://ctan.org/pkg/xcolor

%opening
\title{T2FL Implementation:Mendel, Jerry M., and RI Bob John. "Type-2 fuzzy sets made simple." IEEE Transactions on fuzzy systems 10.2 (2002): 117-127.}
\author{Carmel Gafa}

\begin{document}

\maketitle

\begin{abstract}

\end{abstract}

\section{Type-2 fuzzy set definition}

\begin{tabular}{|l|}
	\hline 
	\lstset{language=Python}
	\lstset{basicstyle=\scriptsize}
	\begin{lstlisting}
	from type2fuzzy import GeneralType2FuzzySet
	
	'''
	Example 1 : definition of the general type-2 fuzzy set
	'''
	
	gt2fs_rep =   ''' (0.9/0 + 0.8/0.2+ 0.7/0.4 + 0.6/0.6 + 0.5/0.8)/1
	+(0.5/0 + 0.35/0.2 + 0.35/0.4 + 0.2/0.6 + 0.5/0.8)/2
	+(0.35/0.6 + 0.35/0.8)/3
	+(0.1/0 + 0.35/0.2 + 0.5/0.4 + 0.1/0.6 + 0.35/0.8)/4
	+(0.35/0 + 0.5/0.2 + 0.1/0.4 + 0.2/0.6 + 0.2/0.8)/5'''
	
	# create set
	print('\nSet representation:')
	gt2fs = GeneralType2FuzzySet.from_representation(gt2fs_rep)
	print(gt2fs)
	\end{lstlisting}
	
	\\
	\hline
	
	Set representation:
	(0.9000 / 0.0000 + 0.8000 / 0.2000 + 0.7000 / 0.4000 + \\
	0.6000 / 0.6000 + 0.5000 / 0.8000) / 1.0000\\
	 + (0.5000 / 0.0000 + 0.3500 / 0.2000 + 0.3500 / 0.4000 + \\
	 0.2000 / 0.6000 + 0.5000 / 0.8000) / 2.0000 \\
	 + (0.3500 / 0.6000 + 0.3500 / 0.8000) / 3.0000 + (0.1000 / 0.0000\\
	  + 0.3500 / 0.2000 + 0.5000 / 0.4000 +\\
	  0.1000 / 0.6000 + 0.3500 / 0.8000) / 4.0000 + \\
	  (0.3500 / 0.0000 + 0.5000 / 0.2000 + 0.1000 / 0.4000 + \\
	  0.2000 / 0.6000 + 0.2000 / 0.8000) / 5.0000
	
	\\ 
	\hline 
\end{tabular} 


\section{Verticalk Slice}


\begin{tabular}{|l|}
	\hline 
	\lstset{language=Python}
	\lstset{basicstyle=\scriptsize}
	\begin{lstlisting}
# different ways to get vertical slice
print('mu_a_tilde(',1,')= ', gt2fs.vertical_slice(1))
print('mu_a_tilde(',2,')= ', gt2fs[2])
print('mu_a_tilde(',3,')= ', gt2fs.vertical_slice(3))
print('mu_a_tilde(',4,')= ', gt2fs[4])
	\end{lstlisting}
	
	\\
	\hline
	
mu( 1 )=  \\
0.900/0.000 + 0.800/0.200 + 0.700/0.400 + 0.600/0.600 + 0.500/0.800\\
mu( 2 )=  \\
0.500/0.000 + 0.350/0.200 + 0.350/0.400 + 0.200/0.600 + 0.500/0.800\\
mu( 3 )=  \\
0.350/0.600 + 0.350/0.800\\
mu( 4 )=  \\
0.100/0.000 + 0.350/0.200 + 0.500/0.400 + 0.100/0.600 + 0.350/0.800\\

	\\ 
	\hline 
\end{tabular} 








\end{document}








